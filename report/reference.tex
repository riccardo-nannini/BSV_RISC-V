\documentclass[12pt,oneside,a4paper]{article}

\usepackage[backend=biber,style=numeric]{biblatex}
\usepackage{xcolor}
\usepackage{todonotes}
\usepackage{amsmath}
\usepackage{multicol}
\usepackage{caption}
\usepackage{hyperref}
\usepackage{graphicx}
\usepackage{listings}

\lstdefinestyle{bsvstyle}{
	backgroundcolor=\color{backcolour},   
	numberstyle=\tiny\color{codegray},
	basicstyle=\ttfamily\footnotesize,
	breakatwhitespace=false,         
	breaklines=true,                 
	captionpos=b,                    
	keepspaces=true,                 
	numbers=left,                    
	numbersep=5pt,                  
	showspaces=false,                
	showstringspaces=false,
	showtabs=false,                  
	tabsize=2
}

\lstdefinelanguage{BSV}{
	keywords={typeof, new, true, false, catch, function, return, null, catch, switch, var, if, in, while, do, else, case, break, globaldata, procesure, AND},
	keywordstyle=\color{blue}\bfseries,
	ndkeywords={class, export, boolean, throw, implements, import, this},
	ndkeywordstyle=\color{darkgray}\bfseries,
	identifierstyle=\color{black},
	sensitive=false,
	comment=[l]{//},
	morecomment=[s]{/*}{*/},
	commentstyle=\color{purple}\ttfamily,
	stringstyle=\color{red}\ttfamily,
	morestring=[b]',
	morestring=[b]"
}

\definecolor{codegreen}{rgb}{0,0.6,0}
\definecolor{codegray}{rgb}{0.5,0.5,0.5}
\definecolor{codepurple}{rgb}{0.58,7,0.82}
\definecolor{backcolour}{RGB}{232, 232, 232}

\lstset{
	language=BSV,
	style=bsvstyle
}
\DeclareCaptionFormat{listing}{\rule{\dimexpr\textwidth\relax}{0.4pt}\par\vskip1pt#1#2#3}
\captionsetup[lstlisting]{format=listing,singlelinecheck=false, margin=0pt,labelsep=space,labelfont=bf}

\usepackage{booktabs}
\usepackage[noabbrev,capitalise]{cleveref}
\crefname{listing}{algorithm}{algorithms}
\Crefname{listing}{Algorithm}{Algorithms}
\renewcommand\lstlistingname{Algorithm}
\def\lstlistingcrefname{Algorithm}
\usepackage{url}

\addbibresource{biblio.bib}

\title{\textbf{RISC-V processors with Bluespec}}

\author{High Performance Processors and System\\ A.Y. 2020/2021\\\\\\\\\textbf{Riccardo Nannini}\\\\\\}

\date{\parbox{\linewidth}{\centering%
		\today\endgraf\bigskip\bigskip\bigskip\bigskip\bigskip
		Tutors: Emanuele Del Sozzo, \endgraf\medskip
		Davide Conficconi \endgraf\bigskip\bigskip
		Professor: Marco Domenico Santambrogio 
		}}
\begin{document}

\begin{titlepage}
	\centering
	\clearpage
	\maketitle
	\thispagestyle{empty}
	\vspace*{1cm}
	\vfill
	\centering
	\includegraphics{logo_polimi.png}\includegraphics{logo_NECST.png}
\end{titlepage}

\tableofcontents
\newpage

\begin{abstract}
Bluespec System Verilog (\textbf{BSV}) is a state-of-the-art Hardware Description Language.
Bluespec compilation toolchain (\textbf{BSC}) has been recently released as open source \cite{bsc}. The goal of the project was investigating the potentiality of said toolchain implementing different \textbf{RISC-V} processors of increasing complexity.\\
\end{abstract}



\section{RISC-V} \label{sec:sec1}
In the following, \cref{eq:exeq} shows an example of equation centered within the page.

\begin{equation}\label{eq:exeq}
%
\mbox{maximize} \sum_{i=17}^{31} \sum_{j=i+1}^{32} [ x_{i,j} \times s_{i,j} + (1 - x_{i,j}) \times d_{i,j} ]
%
\end{equation}

To type any mathematical expression in the text without breaking the line, you can surround it with the \$ symbol, for example to refer to $i$ and to $ \sum_{j=i+1}^{32} [ x_{i,j} \times s_{i,j} + (1 - x_{i,j}) \times d_{i,j} ] $.

If you need to show code snippets, you can use the \emph{listing} environment, as in the following example. As for the other elements, you can refer to a listing through its label as in \cref{list:exlistings}. Remember to make your code well readable by indenting it and using concise pseudo-code snippets, without pasting your own code as it is (unless it is REALLY expressive and short).

\begin{lstlisting}[label={list:exlistings},caption={Example of code snippet}]
globaldata: list_head buddies[MAX_ORDER][MAX_COLORS]

procedure InsertBuddy(buddy b, order ord)
	buddy twin
	mcolor mcol
	
	mcol = Mcolor(b, ord)
	twin = GetTwinBuddy(b,ord)	
	if ord < MAX_ORDER-1 AND BuddyIsFree(twin)
		RemoveFromList(buddies[ord][Mcolor(twin,ord)])
		b = CoalesceBuddy(b, twin, ord)
		InsertBuddy(b, ord+1)
		return
	else
		InsertHead(buddies[ord][mcol],b)
end procedure
\end{lstlisting}

\subsection{Subsection 1} \label{sec:sub1}
This is the way to refer to , and similarly for \cref{sec:sec1}.

You will notice \LaTeX~ freely moves elements like figures and tables around the page, and often in the pages around the current paragraph. In particular, \LaTeX~ always places these elements at the bottom or top of the page (otherwise instructed): this choice obeys to the main typesetting guidelines, and should work well most of the times. You should not force a specific position for these elements, and keep in mind that \textit{\LaTeX~ most of the time is right} (it is its job to do lay out elements, not yours). If you need to move an element, move its \LaTeX~ code up or down.

\subsection{Subsection 2} \label{sec:sub2}
\Cref{tab:example} provides an example of a table. According to many people, this table style (without vertical lines separating columns) is the most elegant and clean possible; to set this tables style, this document adds the \verb|\usepackage{booktabs}| directive at the beginning. In the \LaTeX~ code, you can notice that an ampersand (\$) separates columns and a double backslash (\verb|\\|) moves to a new line.
\begin{table}
	\caption{Table title (without stop!)}
	\centering
	\begin{tabular}{ccl}
		\toprule
		\textbf{label0} & \textbf{label1} & \textbf{label2} \\
		\midrule
		row 0, col 0 & row 0, col 1 & row 0, col 2 \\
		row 1, col 0 & row 1, col 1 & row 1, col 2 \\ 
		row 2, col 0 & row 2, col 1 & row 2, col 2 \\ 
		row 3, col 0 & row 3, col 1 & row 3, col 2 \\
		\bottomrule
	\end{tabular}
	\label{tab:example}
\end{table}

Since tables in \LaTeX~ are verbose, you should:
\begin{itemize}
\item place them in a specific file, to be included with a \verb|\input{filename}| directive
\item for large tables, fill them on applications or websites like \url{https://www.tablesgenerator.com/}, then copy their code and paste it in the dedicated file; finally, you can customize the style from the \LaTeX~ code
\end{itemize}

Here you can see the same table as before but included from an external file \textit{table.tex}: the result is the same.
%
\input{table}

\printbibliography

\end{document}


